% ============================================================================
% SAMPLE DOCUMENT USING ENHANCED ATRAJIT (ZETA) THEME
% ============================================================================
% This template demonstrates all the new features and macros available
% in the enhanced zeta theme with professional styling
% ============================================================================

\documentclass[12pt, a4paper]{article}

% ============================================================================
% PACKAGE SELECTION
% ============================================================================

% Enhanced Zeta Theme (Blue colors, zeta symbols, professional styling)
\usepackage{atrajit}

% ============================================================================
% DOCUMENT METADATA
% ============================================================================

\title{\textbf{Enhanced Zeta Theme Guide}\\
       \large A Comprehensive Demonstration of Professional Features}
\author{Atrajit Sarkar}
\date{\today}

% Bibliography file (if needed)
% \addbibresource{references.bib}

% ============================================================================
% DOCUMENT CONTENT
% ============================================================================

\begin{document}

\maketitle

\begin{abstract}
This document demonstrates the enhanced \texttt{atrajit} (zeta) theme with professional features including keyboxes, elegant section breaks, mathematical set shortcuts, text highlighting, and more. All features maintain the classic blue color palette with zeta-themed decorations.
\end{abstract}

\tableofcontents
\pagebreak


% ============================================================================
\section{Introduction to Enhanced Features}
% ============================================================================

The enhanced \texttt{atrajit} package provides a complete professional look for mathematical documents with the following key features:

\begin{itemize}
    \item Professional blue color palette (zetaPrimary, zetaSecondary, zetaAccent)
    \item Elegant decorative elements throughout
    \item Enhanced theorem environments
    \item Keybox highlighting for important formulas
    \item Mathematical set notation shortcuts
    \item Text highlighting macros
    \item Reference shortcuts
\end{itemize}

\subsection{Color Scheme}

The theme uses a sophisticated academic blue palette:
\begin{itemize}
    \item \textcolor{zetaPrimary}{\textbf{Primary:}} Deep Blue for main elements
    \item \textcolor{zetaSecondary}{\textbf{Secondary:}} Royal Blue for accents
    \item \textcolor{zetaAccent}{\textbf{Accent:}} Cornflower Blue for highlights
\end{itemize}

% ============================================================================
\section{Mathematical Set Shortcuts}
% ============================================================================

\subsection{Common Sets}

The package provides convenient shortcuts for standard mathematical sets:

\keybox{
$$
\begin{array}{ll}
    \RR \text{ (Real numbers)}, \quad
\QQ \text{ (Rationals)}, \quad
\ZZ \text{ (Integers)}, \quad
\NN \text{ (Natural numbers)},\\
\CC \text{ (Complex numbers)}
\end{array}
$$
}

\textbf{Usage:}
\begin{verbatim}
\RR    Real numbers
\QQ    Rational numbers
\ZZ    Integers
\NN    Natural numbers
\CC    Complex numbers
\end{verbatim}

\sectionbreak

% ============================================================================
\section{Keybox Feature}
% ============================================================================

\subsection{Highlighting Important Formulas}

Use \verb|\keybox{...}| to highlight key equations and important content:

\keybox{
\textbf{Euler's Identity:}
$$
e^{i\pi} + 1 = 0
$$
This beautiful equation connects five fundamental mathematical constants.
}

\subsection{Multiple Equations in Keybox}

\keybox{
\begin{equation}
    \zeta(s) = \sum_{n=1}^{\infty} \frac{1}{n^s}
\end{equation}
\begin{equation}
    \zeta(2) = \frac{\pi^2}{6}
\end{equation}
}

\sectionbreak

% ============================================================================
\section{Text Highlighting Macros}
% ============================================================================

\subsection{Mathematical Terms}

Emphasize important mathematical terms using \verb|\mathterm{...}|:

The \mathterm{eigenvalue} of a matrix is a scalar that satisfies the characteristic equation. The concept of \mathterm{convergence} is fundamental in analysis.

\subsection{Definition Terms}

Use \verb|\defterm{...}| for defining new concepts:

A \defterm{metric space} is a set equipped with a distance function. We say a sequence is \defterm{Cauchy} if its terms become arbitrarily close.

\subsection{Code and Notation}

Use \verb|\code{...}| for inline code or special notation:

The function \code{zeta(s)} computes the Riemann zeta function. Use \code{integrate(f, a, b)} for numerical integration.

\sectionbreak

% ============================================================================
\section{Theorem Environments}
% ============================================================================

\subsection{Standard Theorems}

\begin{theorem}[Fundamental Theorem of Calculus]
Let $f$ be continuous on $[a,b]$. Then
$$
\int_a^b f(x)\,dx = F(b) - F(a)
$$
where $F$ is any antiderivative of $f$.
\end{theorem}

\begin{proof}
This follows from the definition of the definite integral.
\end{proof}

\begin{lemma}
Every convergent sequence in $\RR^n$ is bounded.
\end{lemma}

\begin{proposition}
The intersection of two open sets is open.
\end{proposition}

\begin{corollary}
The set of irrational numbers is uncountable.
\end{corollary}

\subsection{Definitions and Examples}

\begin{definition}[Topology]
A \defterm{topology} on a set $X$ is a collection $\mathcal{T}$ of subsets of $X$ satisfying:
\begin{enumerate}
    \item $\emptyset, X \in \mathcal{T}$
    \item Arbitrary unions of sets in $\mathcal{T}$ are in $\mathcal{T}$
    \item Finite intersections of sets in $\mathcal{T}$ are in $\mathcal{T}$
\end{enumerate}
\end{definition}

\begin{example}
The standard topology on $\RR$ is generated by open intervals $(a,b)$.
\end{example}

\begin{remark}
Not every collection of subsets forms a topology.
\end{remark}

\begin{note}
This is a useful note about the concept being discussed.
\end{note}

\sectionbreak

% ============================================================================
\section{Custom Markers}
% ============================================================================

\subsection{TODO Items}

Use \verb|\todotea| for tasks that need attention:

\todotea\ Complete the proof of convergence for the series.

\todotea\ Add references to recent papers on this topic.

\subsection{Important Notes}

Use \verb|\notebell| for important notes:

\notebell\ This assumption is crucial for the proof to work.

\subsection{Critical Information}

Use \verb|\importantmark| for warnings or critical information:

\importantmark\ The function must be continuous for this theorem to apply.

\sectionbreak

% ============================================================================
\section{Reference Shortcuts}
% ============================================================================

\subsection{Theorem References}

The package provides convenient shortcuts for referencing theorems:

\begin{theorem}\label{thm:main}
This is an important theorem.
\end{theorem}

\begin{lemma}\label{lem:key}
This is a key lemma.
\end{lemma}

Later in the document, you can reference them using shortcuts:

\textbf{Usage:}
\begin{verbatim}
\thmref{thm:main}    % Theorem~\ref{thm:main}
\lemref{lem:key}     % Lemma~\ref{lem:key}
\propref{prop:basic} % Proposition~\ref{prop:basic}
\corref{cor:result}  % Corollary~\ref{cor:result}
\defref{def:space}   % Definition~\ref{def:space}
\exref{ex:simple}    % Example~\ref{ex:simple}
\end{verbatim}

\textbf{Examples:} \thmref{thm:main} is important, and \lemref{lem:key} supports it. 

\sectionbreak

% ============================================================================
\section{Section Breaks}
% ============================================================================

Use \verb|\sectionbreak| to add elegant ornamental dividers between sections or subsections. This creates a professional visual separation with:

\begin{itemize}
    \item Ornamental horizontal lines
    \item Diamond symbols
    \item Central zeta circle badge
    \item Decorative dots at the ends
\end{itemize}

The section break appears throughout this document between major sections.

% ============================================================================
\section{Advanced Examples}
% ============================================================================

\subsection{Complex Mathematical Content}

\keybox{
\textbf{Multiple Zeta Values:}
$$
\zeta(k_1, k_2, \ldots, k_r) = \sum_{n_1 > n_2 > \cdots > n_r > 0} \frac{1}{n_1^{k_1} n_2^{k_2} \cdots n_r^{k_r}}
$$
}

\begin{theorem}[Stuffle Product]
For positive integers $k_1, k_2$, we have:
$$
\zeta(k_1)\zeta(k_2) = \zeta(k_1,k_2) + \zeta(k_2,k_1) + \zeta(k_1+k_2)
$$
\end{theorem}

\subsection{Using Multiple Features}

The \mathterm{Riemann zeta function} $\zeta(s)$ is defined for $s \in \CC$ with $\Re(s) > 1$ by:

\keybox{
$$
\zeta(s) = \sum_{n=1}^{\infty} \frac{1}{n^s} = \prod_{p \text{ prime}} \frac{1}{1-p^{-s}}
$$
}

\notebell\ The product formula shows the deep connection between $\zeta(s)$ and prime numbers.

\begin{definition}[Analytic Continuation]
The \defterm{analytic continuation} of $\zeta(s)$ extends the function to all $s \in \CC \setminus \{1\}$.
\end{definition}

\sectionbreak

% ============================================================================
\section{Equation Numbering}
% ============================================================================

\subsection{Default Numbering}

By default, equations are numbered by section:

\begin{equation}
    \int_{-\infty}^{\infty} e^{-x^2} dx = \sqrt{\pi}
\end{equation}

\subsection{Subsection Numbering}

Use \verb|\useSubsectionEqNum| to number equations by subsection (as shown in main.tex).

\subsection{Subsubsection Numbering}

Use \verb|\useSubsubsectionEqNum| to number equations by subsubsection:

\begin{verbatim}
\useSubsubsectionEqNum
\end{verbatim}

% ============================================================================
\section{Hyperlink Styling}
% ============================================================================

The theme includes professional hyperlink styling with colored links:

\begin{itemize}
    \item Internal links (sections, equations): \textcolor{zetaPrimary}{Deep Blue}
    \item Citations: \textcolor{zetaSecondary}{Royal Blue}
    \item URLs: \textcolor{zetaAccent}{Cornflower Blue}
\end{itemize}

All links are properly configured with bookmarks and navigation features.

\sectionbreak

% ============================================================================
\section{Best Practices}
% ============================================================================

\subsection{When to Use Keyboxes}

Use keyboxes for:
\begin{enumerate}
    \item Main theorems and results
    \item Key formulas that should stand out
    \item Important definitions
    \item Summary equations
\end{enumerate}

\subsection{Section Organization}

Use \verb|\sectionbreak| between:
\begin{itemize}
    \item Major sections or exercises
    \item Different topics within a section
    \item Before and after important results
\end{itemize}

\subsection{Text Highlighting}

\begin{itemize}
    \item Use \verb|\mathterm{}| for technical mathematical terms
    \item Use \verb|\defterm{}| when first introducing a concept
    \item Use \verb|\code{}| for notation, functions, or algorithms
\end{itemize}

% ============================================================================
\section{Conclusion}
% ============================================================================

The enhanced \texttt{atrajit} (zeta) theme provides a complete professional toolkit for mathematical documents. Key features include:

\keybox{
\textbf{Summary of Features:}
\begin{itemize}
    \item Professional blue color palette
    \item Keybox highlighting: \texttt{\textbackslash keybox\{...\}}
    \item Mathematical sets: \texttt{\textbackslash RR}, \texttt{\textbackslash QQ}, \texttt{\textbackslash ZZ}, \texttt{\textbackslash NN}, \texttt{\textbackslash CC}
    \item Text highlighting: \texttt{\textbackslash mathterm\{\}}, \texttt{\textbackslash defterm\{\}}, \texttt{\textbackslash code\{\}}
    \item Custom markers: \texttt{\textbackslash todotea}, \texttt{\textbackslash notebell}, \texttt{\textbackslash importantmark}
    \item Reference shortcuts: \texttt{\textbackslash thmref\{\}}, \texttt{\textbackslash lemref\{\}}, etc.
    \item Elegant section breaks: \texttt{\textbackslash sectionbreak}
\end{itemize}
}

\sectionbreak

\begin{center}
\textit{Happy typesetting with the enhanced Zeta theme!}
\end{center}

% ============================================================================
% BIBLIOGRAPHY (if needed)
% ============================================================================
% \printbibliography

\end{document}
